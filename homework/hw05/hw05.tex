\documentclass{article}
\usepackage[margin=1in]{geometry}
\usepackage{hyperref}
\usepackage{graphicx}
\usepackage{fancyvrb}


\newcommand{\myitem}{\paragraph}
\begin{document}

\myitem{CSCI 342, Fall 2017, Homework \# 5}

\myitem{Due date:}  Friday, December 1, midnight.  ZIP all files
together, including images and other resources.

\myitem{Instructions:}

\begin{itemize}
\item
First, follow the instructions found here to build a popup penguin game
using just HTML and CSS:

\centerline{\url{https://googlecreativelab.github.io/coder-projects/projects/pop_up_penguins/}}

\item Now modify the program and add Javascript (Vanilla or jQuery)
  to accomplish the first two ``Bonus Rounds'' as follows.

\item For this assignment, use the Javascript DOM as much as possible.
  Avoid printing HTML tags with your Javascript, and instead create
  objects and attach them as children to existing objects in the page.

\item Use Javascript arrays and for-loops to populate the page with
  penguins instead of hard coding every penguin in HTML.  This will
  allow you to put any number of penguins on the page.

\item Choose the position of the Yeti with a random number.  Every
  time you play the game it will be in a different position.
  
\item Add a text input at the top of the page to choose the number of
  penguins (+ Yeti) in the game.  Validate that this is an integer between 2
  and 64.  As soon as the user enters a number, the new game should
  load (no submit button).  The currently chosen number of penguins (+
  Yeti)  should remain in the text box while the game is played.

\item Make the penguins stay up when clicked, and disappear when the
  Yeti is awoken.

\item After the Alert panel is closed, focus should go back to the
  number of penguins (+ Yeti) box, so the user can simply hit return
  to play a new game (or change the number).

\item Optional:  add real squawks, chirps, and roars.  

\end{itemize}





\end{document}
